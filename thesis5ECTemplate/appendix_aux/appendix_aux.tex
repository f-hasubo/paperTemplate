% \chapter{DNNの人工データに対する予測結果}
% \label{chap:artificial}

% %----------------------------------------------
% \section{接線不等式}
% %----------------------------------------------

% \begin{lemm} \label{lem:aux:sessen} (接線不等式)
% $f(x)$が凹関数であるとき,以下の不等式が成立する.
% \begin{align}
% f(x) \leq f'(\bar{x}) (x - \bar{x}) + f(\bar{x})
% \end{align}
% 不等式中の等号が成立するための条件は$x = \bar{x}$である.
% \end{lemm}

% %----------------------------------------------
% \section{Jensenの不等式}
% %----------------------------------------------

% \begin{lemm} \label{lem:aux:jensen} (Jensenの不等式)
% $\alpha_{i} > 0$を,$\sum_{i} \alpha_{i} = 1$を満たす補助変数とする.
% 関数$f(x)$が凸関数であるとき,$x_{i}\ (i = 1,\ldots,I)$に対して以下の不等式が成立する.
% \begin{align}
% f \left( \sum_{i=1}^{I} \alpha_{i} x_{i} \right) \leq \sum_{i=1}^{I} \alpha_{i} f(x_{i})
% \end{align}
% $f(x)$が狭義凸関数であるとき,不等式中の等号が成立するための条件は
% $x_1 = \cdots = x_i = \cdots = x_I$である.
% \end{lemm}
