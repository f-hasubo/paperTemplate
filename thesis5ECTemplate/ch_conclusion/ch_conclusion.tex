\chapter{結言}
\label{chap:con}

本論文では,FDICAに伴うパーミュテーション問題の解決を目的とし,DNNを用いたパーミュテーション解決法を新たに提案した.
DNNの入力には,ミニ振幅スペクトログラム成分を用いた.テストデータに対してはDNNの入力となる局所時間振幅スペクトログラムをストライド幅に従って,ずらしていくことで時間方向に対して多数決処理を行った.
また,誤差逆伝播の際に,スペクトログラム同士でMSEを行いDNNのモデルを最適化した.
実験結果より,ブロック単位でのパーミュテーション問題に対しては提案手法を用いて正しく並び替えができることを示した.

最後に今後の展望を述べる.
本論文では,DNNを用いた3音源以上にも対応できる新しいパーミュテーション解決手法の可能性に注目しており,基礎的な実験を行ってきた.
ただ,実際に3音源以上での実験は行っていない.今回行った2音源での実験の拡張版として,今後は3音源以上に対する実験も行っていきたい.
また,リアルタイムでの音源分離に適用する場合は,DNNモデル及び損失取得時の計算アルゴリズムを改良する必要がある.
