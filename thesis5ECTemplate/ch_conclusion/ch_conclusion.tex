\chapter{結言}
\label{chap:con}

本論文では,FDICAに伴うパーミュテーション問題の解決を目的とし,深層パーミュテーション解決法を新たに提案した.
DNNの入力には,正規化した分離信号から局所時間振幅スペクトログラム成分を抽出した値を用いた.
DNNの出力には,softmax関数を使用し確率値を出力する.この確率値は,各音源の成分である確率を意味する.
DNNの出力である確率値を用いて,推定パーミュテーション行列を作成し分離信号の並び替えを行った.
損失関数にはMSEを用い,推定分離信号と完全分離信号のスペクトログラム間で損失を求め誤差逆伝搬を行った.
テストデータに対してはDNNの入力となる局所時間振幅スペクトログラムをストライド幅に従ってずらしていくことで,時間方向に対して多数決処理を行った.
実験結果より,ブロック単位でのパーミュテーション問題に対しては提案手法を用いて正しく並び替えができることを示した.

最後に今後の展望を述べる.
本論文では,深層パーミュテーション解決手法の可能性に注目しており,基礎的な実験を行ってきた.
本実験では,学習データと検証データの音源に同じスペクトログラムを用いて実験を行っており,未だに音源の時間周波数構造に対する汎化性能は獲得できていない.
この問題を解決するためには,学習データに多数の音声及び音楽信号を用意し,大量のデータをDNNに学習させる必要がある.
また,更なる精度向上のために,DNNの構造としてMLPを用いるのではなく,双方向再帰型DNNを使用することを検討している.
さらには,2音源の実験の拡張版として3音源以上に対する実験も行う必要がある.
